\section*{M. Ternarian Weights}

This task asks for us to find two sets of powers of three that have a difference of $x$.

We notice that if we consider $x$ modulo 3, then there are obviously 3 cases,
0, 1 or 2.

In the case of 0, we notice that we can divide the entire problem through by 3,
and we are now trying to solve the problem ``find two sets of powers of powers
of three that have a difference of $\frac x 3$''.

In the case of 1, we can counter this by placing a weight of $3^0 = 1$ pounds
on the right scale, and now we have a similar case to the above where we're
trying to solve for $\frac x 3$.

In the case of 2, we add a weight of $3^0 = 1$ points to the left scale,
rounding the weight difference to the next multiple of three, and thus leaving
us with the same problem for $\frac {x + 1} 3$.

This leaves us with a nice recursive algorithm that runs in $O(\log x)$. During
the competition we chose to implement this iteratively since this happened to
be the way that we thought of it at the time. The idea of implementing this
recursively only came after the competition, and the implementation would be
more or less the same.

One interesting thing to note with this algorithm is that it almost acts as a
proof that the solution will be unique. Given some $x$, our only way to handle
the least significant digit (base 3) is as described in the algorithm. We need
to handle it, otherwise we can not have a difference of exactly $x$.
