\section*{A. What Does the Fox Say?}

The task asks us to print out a list of ``animal sounds'' with certain ``invalid''
animal sounds filtered out. This task is rather straightforward, and the bounds
are such that we can naively iterate through the invalid sounds for each string
in the list we are to filter, and print the word in the list accordingly.

Perhaps the most interesting part of this task is reading in the input, as we
are not told the number of words in the list. There are a few ways of doing
this. The method presented in the source relies upon the word list having no
trailing spaces. We scan in each string and the character immediately following
it, stopping when we find a string followed by a newline. Other methods include
reading in the entire line and parsing each word based on whitespace, as well
as using the {\tt stringstream} library to parse the line.

The same can be done for parsing the list of invalid sounds. Alternatively,
it's possible to use strchr to take advantage of strchr since the index of the
word is known, as we have done.

In order to detect invalid words as we iterating through the word, instead of
the naive iteration through the invalid words, we can use an STL {\tt set}. If
using C, we could implement a prefix tree to a similar effect.

The algorithm is $O(L|S|\log|I|)$, where $|S|$ is the size of the word list,
$|I|$ the size of the invalid word list and $L$ the maximum length of the
strings. The naive algorithm would be $O(L|S||I|)$, which (as mentioned above)
would be allowed with the given bounds.
