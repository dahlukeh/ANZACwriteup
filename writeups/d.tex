\section*{D. Shopping Malls}

This task provides us with a representation of a graph, with edges being of one
of four types \emph{walking}, \emph{stairs}, \emph{lift} or \emph{escalator}.
We are asked to determine the path between several pairs of vertices with shortest
walking distance.

We first note that for each method of travel, we have a method for determining
the walking distance for that method. In particular, given some edge we can
determine the walking distance for that edge regardless of other edges used in
a path containing this edge. Note that this may not be the case if, say, we had
another edge type \emph{fitness door} that only allowed access after a certain
distance had been travelled prior to reaching one of its endpoints. All methods
of travel are bidirectional, however some methods have different weights for
each direction, so we treat the entire graph as directed so as to allow this
asymmetry to be represented.

We notice that we are guaranteed that there is a path from every vertex to
every other vertex, and so we need not handle that case. We are also guaranteed
that there can be no implicit edges or vertices (like those that might be
formed from lifts visiting intermediate floors), so we can take the edge list
as is.

Given that all edges have a fixed weight, we find that the task can be solved
with a simple application of any shortest path algorithm that would work on a
directed weighted graph. Unfortunately, in this task we are required to provide
the path itself for each query, rather than just the length of the path. We
will describe two methods for computing this.

\subsection*{With Dijkstra's algorithm}

We notice that even a naive $O(N^2)$ application of Dijsktra's algorithm for
each query is acceptable given the bounds.

In order to compute the path itself using Dijkstra's algorithm, we can maintain
a lookup table for each vertex $v_i$ that contains the \emph{previous} vertex
in the shortest path from the source vertex to $v_i$. This would be maintained
in the same manner as the table of best known distances to each vertex.

When we have completed the algorithm, we are left with a table that we can
\emph{backtrack} through. That is, we look at our target vertex and repeatedly
follow the vertices lookup table backwards, reforming the path backwards from
the target to the source. Note that if the graph were symmetric, we could have
instead calculated the shortest path from the target to the source, removing
the need to reverse our resultant path.

If we were to do this for each query, we would have an algorithm that runs in
$O(N^2Q)$ time, which is sufficient for the given bounds.

Note that if we had simple interesting edges, such as the fitness door
mentioned above, this algorithm would still work.

\subsection*{With the Floyd-Warshall algorithm}

We also notice that a $O(N^3)$ precalculation of the shortest paths for all
pairs of vertices would also be sufficient, as we can then answer all queries
in constant-ish time. We can't answer queries in constant time as we have to
provide the path, which is of $O(N)$ length. This would still be sufficient,
though.

Again, having to deal with providing paths makes this implementation of the
Floyd-Warshall algorithm somewhat less trivial. We use a similar technique to
the one used in Dijkstra's algorithm in that we will keep track of the pivot
vertex that we used to get the shortest path between all pairs of vertices.
This will be maintained in a similar manner to how the table of best distances
is maintained and updated.

Once we have calculated the shortest path between all pairs of vertices, to
answer a query requires recursively (or otherwise) rebuilding the path based
off the pivots used. We define a function $g(a, b)$ to be sequence of vertices
that we visit in a shortest path from $a$ to $b$. To prevent duplication of
vertices while computing this, we will define the sequence of vertices such
that it must contain $a$ as the first element, and will {\bf not} contain $b$
as it's last element. The function is inclusive-exclusive.

If we maintained our table of pivots correctly, we should have the following
definition for $g$:

\[g(a, b) = \left\{
  \begin{array}{lr}
      g(a, pivot_{a, b}) \| g(pivot_{a, b}, b) & \text{if } pivot_{a, b} \text{is defined} \\
      {[}a{]} & \text{otherwise}
  \end{array}
\right.
\]

\noindent
where $\|$ represents the concatenation of sequences.

We then print out this sequence suffixed by the target vertex, since out
definition of $g$ omits it.

Note here that the existence of a path between vertices $a$ and $b$ would imply
that we've calculated some pivot for it. This is only not the case if there
exists a direct edge from $a$ to $b$, in which case our definition of $g$ would
have us simply return $a$.

This gives us a $O(N^3 + NQ)$ solution.

\subsection*{Some extra thoughts}

During the contest, we had to choose between the two methods detailed above. We
almost immediately went for the second, since an implementation of
Floyd-Warshall's algorithm is extremely short and easy to write up. It's the
less ``general'' of the two, since we cannot handle edges that might (for
example) rely upon the length of the path so far, but this did not matter in
this case.
